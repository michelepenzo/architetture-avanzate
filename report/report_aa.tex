\documentclass[]{IEEEtran}

\title{Sparse Matrix Transposition for GPUs}
\author{Massimiliano Incudini - VR433300\\Michele Penzo - VR439232}

\usepackage{graphicx}
\usepackage[export]{adjustbox}
\usepackage[italian]{babel}

\begin{document}
\maketitle

\begin{abstract}
	L'obiettivo principale di questo progetto è stato quello di implementare alcune metodologie proposte per effettuare \textit{Sparse Matrix Transposition} su \textit{Gpu}.
	Sono stati analizzati alcuni algoritmi, descritti in sezione~\ref{metodologie}, partendo dall'algoritmo seriale, passando a cuSPARSE per finire con l'implementazione degli algoritmi descritti in~\cite{parallelTrans}.
	Infine vengono esposti i risultati e le conclusioni tratte.
\end{abstract}


\section{Introduzione}
% todo
% parlare del csr
% spiegare lavoro, come è strutturato il tutto, test ...

 
\section{Metodologie analizzate}
\label{metodologie}
	In questa sezione vengono spiegate ed evidenziate le differenze tra le varie metodologie analizzate. 
		
	\subsection{Trasposta seriale}
	% todo
	
	\subsection{Nvidia cuSPARSE}
	Questo toolkit è implementato all'interno nelle librerie NVIDIA CUDA runtime. Le routine delle librerie vengono utilizzate per le operazioni tra vettori e matrici che sono rappresentate tramite diversi formati. Inoltre mette a disposione operazioni che permettono la conversione attraverso diverse rappresentazioni di matrici, ed inoltre la compressione in formato \textit{csr} che è una delle più usate quando si vuole rappresentare matrici sparse in modo efficiente.\newline	
	Il codice è stato sviluppato basandosi su due versioni di cuSPARSE a causa delle Gpu utilizzate. In fase di compilazione viene quindi controllata la versione usata: $ 9 $ o $ 10 $.\newline
	Nel caso in cui la versione usata sia la $ 10 $ vengono svolti alcuni ulteriori passi, viene effettuata l'allocazione dello spazio necessario e del buffer per il calcolo della trasposta. Per quanto riguarda la versione $ 9 $ invece questi passi non sono necessari.\newline
	Infine viene chiamata la procedura \textbf{cusparseCsr2cscEx2} che effettua il calcolo della trasposta. \newline
	Nello specifico la precedente procedura ...% spiegare come funziona e cosa ritorna

	\subsection{Scan Trans}
	% todo
	
	\subsection{Merge Trans}
	%todo 

\section{Risultati}
% todo
% spiegare come abbiamo fatto i test (che valori), e mostrare tabella con varie dimensioni delle matrici e nnz

\section{Conclusioni}
%todo 



\bibliographystyle{IEEEtran}
\bibliography{biblio}

%\appendix
%Se non avete abbastanza spazio, potete inserire le figure delle EFSM in una  pagina extra, appendice. Un esempio di come potete fare solo le Figure~\ref{fig:grande}, \ref{fig:piccola1}, \ref{fig:piccola2}.

\end{document}