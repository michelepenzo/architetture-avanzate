\documentclass[]{IEEEtran}

\usepackage[italian]{babel}
\usepackage{microtype}
\usepackage{graphicx}
\usepackage{float}
\usepackage[export]{adjustbox}
\usepackage{dirtree}
\usepackage{hyperref}
\usepackage{tikz}
\usepackage{url}
\usepackage{inconsolata}

% tabella dei risultati
\usepackage{array}
\usepackage{siunitx}
\usepackage{booktabs}

\newcommand{\ScanTrans}{\textrm{ScanTrans} }
\newcommand{\MergeTrans}{\textrm{MergeTrans} }
\newcommand{\BlockSize}{\textrm{BLOCK\_SIZE} }
\newcommand{\SplitterDistance}{\textrm{SP\_DIST} }
\newcommand{\cuSPARSE}{\textrm{cuSPARSE} }


\graphicspath{{figures/}} 	

\newcommand*\circled[1]{\tikz[baseline=(char.base)]{\node[shape=circle,draw,inner sep=2pt] (char) {#1};}}

\title{Sparse Matrix Transposition for GPUs}
\author{\begin{tabular}{c c}
    Massimiliano Incudini & VR433300\\
    Michele Penzo & VR439232
\end{tabular}}

\begin{document}
\maketitle

\begin{abstract}
L'op
	L'obiettivo principale di questo progetto è stato quello di implementare alcune metodologie proposte per effettuare \textit{Sparse Matrix Transposition} su \textit{Gpu}.
	Sono stati analizzati alcuni algoritmi, descritti in sezione~\ref{metodologie}, partendo dall'algoritmo seriale, passando a cuSPARSE per finire con l'implementazione degli algoritmi descritti in~\cite{parallelTrans}.
	Infine vengono esposti i risultati e tratte le conclusioni.
\end{abstract}

\begin{figure*}[t]
    \centering
	\includegraphics[scale=0.25]{conceptual_transpose.png}
	\caption{Trasformazione da formato esteso a CSR, oppure CSC}
	\label{first_fig}
\end{figure*}

\section{Introduzione e motivazioni}\label{introduzione}

	% problema e motivazioni
	Sempre più applicazioni computazionali in ambito scientifico necessitano di algoritmi che compiano operazioni applicabili su matrici sparse. Si parla di semplici operazioni di algebra lineare, di moltiplicazione o di calcolo della trasposta come in questo caso.\newline
	Il problema analizzato, quello della trasposizione di matrici, si presta bene al calcolo parallelo per l'esecuzione in maniera più efficiente e veloce. Verranno quindi mostrate le basi per la rappresentazione, i problemi riscontrati durante lo sviluppo e analizzati alcuni algoritmi per il calcolo su \textit{Gpu}.\newline


\section{Rappresentazione delle matrici}
\label{rappresentazione}
	Una matrice sparsa non è altro che una matrice i cui valori sono per la maggior parte uguali a zero. La matrice in formato classico necessita di una quantità di memoria minima di $ m $x$ n $ elementi, ma essendo l'obiettivo quello di lavorare su matrici sparse non è stato necessario e utile memorizzare la matrice in formato denso.\newline
	Per rappresentare in modo efficace le matrici sparse senza troppo utilizzo di memoria sono state quindi introdotte ed utilizzate delle forme di rappresentazione matriciale che permettono il salvataggio di dati utilizzando quantitativi di memoria inferiori.\newline
	Di seguito vengono spiegate le due metodologie da noi utilizzate.
	
	\subsection{Formato Csr}
	\label{csr}
	Il \textit{compressed sparse row} è una rappresentazione di una matrice $ M $ basata su tre array monodimensionali, che rispettivamente contengono:
	\begin{enumerate}
		\item \textit{V}: i valori non zero (\textit{nnz}),
		\item \textit{COL\_INDEX}: gli indici delle colonne dove si trovano gli elementi \textit{nnz},
		\item \textit{ROW\_INDEX}: ha un elemento per ogni riga della matrice e rappresenta l'indice in $ V $ dove comincia la riga data.
	\end{enumerate}
	I primi due array sono di dimensione \textit{nnz}, mentre il terzo array è al massimo di dimensione $ m $.
	
	\subsection{Formato Csc}
	\label{csc}
 	Questa metodologia per la rappresentazione è simile alla precedente citata \textit{Csr}, a differenza che i valori vengono letti prima per colonna. Di conseguenza, un indice di riga viene memorizzato per ogni valore e lo stesso viene fatto per i puntatori di colonna .
 	
	\subsection{Da Csr a Csc}
	\label{csr-to-csc}
 	Per il problema della trasposta di matrice è stato quindi utile introdurre entrambe le rappresentazioni. Infatti, ogni algoritmo  descritto in sezione~\ref{metodologie}, necessita di sei array per effettuare il calcolo della trasposta e dare l'output nella tipologia corretta. Abbiamo quindi:
 	\begin{itemize}
 		\item in input il formato \textit{Csr}: csrRowPtr, csrColIdx, csrVal;
 		\item in output il formato \textit{Csc}: cscColPtr, cscRowIdx, cscVal.	
 	\end{itemize}
 	In base a come vengono create le matrici, se in modo casuale oppure se lette da file, vengono effettutate delle operazioni preliminari descritte dalla procedure in sezione~\ref{procedure} che portano ad ottenere gli array in input e in output nel formato corretto per effettuarne il controllo di correttezza.\newline

\section{Metodologie analizzate}
\label{metodologie}
	In questa sezione vengono spiegate ed evidenziate le differenze tra le varie metodologie analizzate. 
		
	\subsection{Trasposta seriale}
%	La prima metodologia descritta è quella seriale. Sempre a partire dalla rappresentazione in formato \textit{csr} della matrice iniziale l'algoritmo crea un array di elementi, dove per ogni colonna della matrice analizzata conta quanti elementi \textbf{nnz} ci sono. Possiamo definire questo array come un istogramma delle frequenze degli elementi delle colonne. 
	La prima metodologia descritta è quella seriale. Sempre a partire dalla rappresentazione in formato \textit{csr} della matrice iniziale l'algoritmo ottiene i puntatori alle colonne (formato csc) a partire dagli indici di colonna (formato csr). Viene quindi applicato un algoritmo seriale di \textit{prefix\_sum} su questo array, per ottenere i valori corretti di \textbf{cscColPtr}. Infine gli indici di riga e i valori nel nuovo formato \textit{csc} vengono sistemati.\newline
	Questa implementazione servirà come base sulla quale verranno eseguiti i controlli degli algoritmi successivamente implementati.
	
	\subsection{Nvidia cuSPARSE}
	Questo toolkit è implementato all'interno nelle librerie NVIDIA CUDA runtime. Le routine delle librerie vengono utilizzate per operazioni tra vettori e matrici che sono rappresentate tramite diversi formati. Inoltre mette a disposione operazioni che permettono la conversione attraverso diverse rappresentazioni di matrici. Supporta inoltre la compressione in formato \textit{csr} che è una delle più usate quando si vuole rappresentare matrici sparse in modo efficiente.\newline	
	Il codice è stato sviluppato partendo dalla guida \cite{cusparse} ed è diviso in due versioni di cuSPARSE a causa delle Gpu utilizzate. In fase di compilazione viene quindi controllata la versione usata: $ 9 $ o $ 10 $.\newline
	Nel caso in cui la versione usata sia la $ 10 $ vengono svolti alcuni ulteriori passi, come l'allocazione dello spazio necessario per l'esecuzione di cuSparse oltre all'allocazione del buffer per il calcolo della trasposta. Per quanto riguarda la versione $ 9 $ invece questi passi non sono necessari.\newline
	Infine viene chiamata la procedura che effettua il calcolo della trasposta. Nel caso in cui la versione di cuSPARSE sia la $ 10 $ viene richiesto come ulteriore parametro l'algoritmo da utilizzare.\newline
	Dopo essere state eseguite entrambe ritornano i valori ottenuti in formato \textit{csc}.

	\subsection{ScanTrans}
	L'algoritmo considerato prevede di effettuare la trasposta di matrici basandosi sul concetto di scan. Partendo sempre dal presupposto di avere in input una matrice in formato \textit{Csr}, vengono costruiti due array ausiliari:
	\begin{itemize}
		\item inter: array bidimensionale di dimensione $ (nthreads+1) * n $,
		\item intra: array monodimensionale di dimensione massima $ nnz $.
	\end{itemize}
	Ogni riga in \textit{inter} contiene il numero di indici della colonna presi dalla thread i-esima. Mentre ogni elemento in \textit{intra} viene utilizzato per salvare l'offeset relativo alla colonna corrispondente all'elemento nnz preso dalla thread. Dopo aver ottenuto gli istogrammi, viene applicato un \textit{vertical scan} su inter, e una \textit{prefix sum} solamente sull'ultima riga di inter. Infine l'algoritmo calcola l'offset assoluto relativo ad ogni elemento nnz e ritorna il tutto in formato \textit{csc}.\newline
	Tutte le procedure utilizzate in \textit{Scan Trans} si trovano in sezione \ref{procedure}, e vengono eseguite nel seguente ordine:
	\begin{enumerate}
		\item pointers to index: \ref{pnt-to-idx},
		\item index to pointers: \ref{idx-to-pnt},
		\item scan: \ref{scan},
		\item reorder elements: \ref{reoder-elem}.
	\end{enumerate}
	
	\begin{figure}[H]
		\includegraphics[scale=0.6]{scantrans.png}
		\caption{Scan Trans, esempio utilizzato in \cite{parallelTrans}.}
		\label{scantrans}
	\end{figure}
	
	\subsection{MergeTrans}
	L'algoritmo considerato prevede due passi importanti: \textit{sort} e \textit{merge}.
	Inizialmente sono stati creati gli indici di riga a partire dai puntatori delle colonne e su questi ultimi è stato fatto un sort su piccole porzioni di array, mantenendo quindi i vari blocchi disordinati tra di loro ma con gli elementi ordinati. Successivamente è stato utilizzato il merge ricorsivo partendo dai blocchi più piccoli e unendoli in blocchi sempre più grandi. Per funzionare questo processo necessita dell'utilizzo di due buffer di memoria che contengono gli elementi appena ordinati. Infine dai puntatori delle colonne vengono estraxtti gli indici e viene fatta la scan che ritorna il risultato in formato \textit{csc}. \newline
	Anche in questo caso le procedure utilizzate si trovano in sezione \ref{procedure}, e sono ordinatamente eseguite come segue:
	\begin{enumerate}
		\item pointers to index: \ref{pnt-to-idx},
		\item segmented sort: \ref{seg-sort},
		\item segmented merge: \ref{seg-merge},
		\item index to pointers: \ref{idx-to-pnt},
		\item scan: \ref{scan}.
	\end{enumerate}
	
	\begin{figure}[H]
		\includegraphics[scale=0.6]{mergetrans.png}
		\caption{Merge Trans, esempio utilizzato in \cite{parallelTrans}.}
		\label{mergetrans}
	\end{figure}
	
% Inserimento sezione delle procedure

\section{Procedure}\label{procedure}

I due algoritmi \ScanTrans e \MergeTrans vengono scomposti in diversi componenti, ognuno dei quali viene valutato nelle performance e testato separatamente. 

\subsection{Scan}
\label{scan}
Questa operazione prende in input un vettore $A = (a_0, a_1, ..., a_n)$ e ritorna un vettore $B=(I, a_0, a_0 \oplus a_1, ..., a_0 \oplus a_1 \cdots \oplus a_{n-1})$ con $\oplus$ è un'operazione binaria il cui elemento identità è $I$. Nel nostro caso l'operazione è la somma. 

L'algoritmo apparentemente sembra difficile da parallelizzare in quanto il risultato di ogni elemento dipende da tutti i gli elementi precedenti. Diverse soluzioni sono state proposte tra cui l'\emph{algoritmo di Blelloch}. Il suo funzionamento in due fasi è illustrato in Figura~\ref{scan_blelloch} e dettagliatamente discusso in \cite{scan}.

\begin{figure}[H]
	\includegraphics[scale=0.3]{scan.png}
	\caption{Algoritmo di Blelloch}
	\label{scan_blelloch}
\end{figure}

L'implementazione prevede che se l'intero vettore riesce ad essere memorizzato all'interno della shared memory di $N$ elementi, allora possiamo calcolare scan con una singola chiamata a kernel. 

Nel caso questo non sia possibile, l'operazione di scan viene segmentata, applicata separatamente a blocchi di $N$ elementi. Successivamente si mantiene un vettore di somme (vettore degli ultimi elementi del blocco), si applica ricorsivamente \emph{scan} su esso e si sommano gli offset ottenuti all'intero vettore di partenza. 

\subsection{Segmented sort}
\label{seg-sort}
Questa operazione prende in input un vettore di lunghezza $n$ ed un intero $\BlockSize$. Il vettore viene diviso in segmenti di lunghezza $\BlockSize$. Gli elementi di ogni segmento vengono permutati in modo che siano ordinati stabilmente. L'intero segmento deve rientrare nella shared memory.  

\begin{figure}[H]
\centering
	\includegraphics[scale=0.15]{segmented_sort.png}
	\caption{Segmented Sort}
	\label{segmented_sort}
\end{figure}

Una volta caricato il blocco in shared memory, la $i$-esima thread del blocco è incaricata di trovare la posizione corretta dell'$i$-esimo elemento all'interno del segmento, ed assegnarlo a tale posizione. L'algoritmo viene illustrato in Figura~\ref{segmented_sort}. 

L'ordinamento deve essere stabile quindi la posizione dell'$i$-esimo elemento di valore $y$ è dato dal numero di elementi $<y$, sommati al numero degli elementi $=y$ per indici $<i$. 

La dimensione ideale del blocco pari a $128$ elementi (caso interi a 32bit), ed è stata trovata empiricamente:
\begin{figure}[H]
    \centering
    \begin{tabular}{SS}
    \toprule
    \textbf{Thread per blocco} & \textbf{Performance (\si{\milli\second})} \\ \midrule
    64 & 9999.0 \\
    128 & 999.5 \\
    256 & 99.0 \\ \bottomrule
    \end{tabular}
    \caption{Performance su array di $2\cdot 10^7$ elementi}
\end{figure}

\subsection{Merge}
\label{merge}
L'operazione di \emph{merge} trasforma un vettore diviso in segmenti di dimensione $\BlockSize$ nel quale gli elementi ogni segmento sono ordinati, in un vettore diviso in segmenti ordinati di dimensione $2 \BlockSize$, ognuno dei quali è l'unione di una coppia di blocchi contigui. 

Differenziamo il caso in cui il blocco di dimensione $\BlockSize$ rientri o meno nella shared memory. 

\subsection{Merge small}

In questo caso una coppia di blocchi rientra completamente nella shared memory. In modo analogo a quanto fatto per l'operazione di \emph{segmented sort}, abbiamo un numero di thread per blocco pari a $\BlockSize$ nel quale l'$i$-esimo thread è incaricato di calcolare la posizione dell'$i$-esimo elemento. 

In questo caso la posizione dell'$i$-esimo elemento del blocco di sinistra è $i+j$ con $j$ posizione dell'elemento all'interno del blocco di destra, trovato attraverso una ricerca binaria in quanto i blocchi sono ordinati (differentemente da quanto avviene per il segmented sort). Analogamente lo stesso avviene per gli elementi del blocco di destra. 

A parità di valore, gli elementi del blocco di sinistra hanno indice minore di quelli di destra. 

\subsection{Merge big}

\begin{figure}[t]
    \centering
	\includegraphics[scale=0.3]{merge_big.png}
	\caption{Merge big}
	\label{merge_big}
\end{figure}

Questo secondo algoritmo è illustrato in Figura~\ref{merge_big} ed originariamente preso da \cite{mergebig}. Applicando sempre \emph{merge small}, oltre che rinunciare alla shared memory, ogni thread del blocco dovrebbe lavorare su più di un elemento, aumentando la complessità della procedura. 

L'algoritmo proposto funziona nel seguente modo:
\begin{enumerate}
    \item recuperiamo dal vettore degli elementi segnaposto detti \emph{splitter} presi a distanza costante e pari ad $\SplitterDistance$; \\
    per ogni coppia di segmenti da mergiare ho una coppia di blocchi di splitter, trovo la posizione di ogni splitter all'interno del blocco di destra ($A$) e di sinistra ($B$);
    \item applico \emph{merge} ricorsivamente sugli array di splitter;
    \item gli indici associati agli splitter dividono la coppia di blocchi in modo tale da poter effettuare tanti merge indipendenti quanti sono gli splitter. \emph{Ogni merge indipendente considererà al massimo $2\SplitterDistance$ elementi} (numero costante che scegliamo tale che rientri nella shared memory).
\end{enumerate}

\subsection{Istogramma / Index to pointers}
\label{idx-to-pnt}
Dato un vettore $A$ di $n$ elementi compresi tra $0$ ed $m-1$ riceviamo un vettore di $m$ elementi nel quale l'$i$-esima cella contiene la frequenza con cui il valore $i$ è presente in $A$. 

L'algoritmo si sviluppa in due fasi:
\begin{enumerate}
    \item il vettore $A$ viene diviso in $N$ segmenti di lunghezza omogenea, ogni segmento viene processato da un blocco di thread che mantiene l'istogramma parziale;
    \item gli istogrammi parziali vengono poi uniti attraverso un'operazione di prefix scan che si "sviluppa in verticale``. Tale operazione può essere ottenuta:
    \begin{itemize}
        \item trasponendo la matrice degli istogrammi parziali ed applicando $N$ volte \emph{prefix sum}; oppure
        \item attraverso $N$ blocchi di thread che attraversano sequenzialmente il vettore colonna degli istogrammi parziali.
    \end{itemize}
    \item si applica \emph{prefix sum} sul risultato dell'operazione precedente.
\end{enumerate}

L'algoritmo è illustrato in Figura~\ref{index_to_pointers}. 

\begin{figure}[t]
    \centering
	\includegraphics[scale=0.22]{index_to_pointers.png}
	\caption{Index to pointers}
	\label{index_to_pointers}
\end{figure}

\subsection{Pointers to index}
\label{pnt-to-idx}
Operazione inversa di \emph{index to pointers}. Il vettore risultante è ordinato. Può essere implementato assegnando ad ogni blocco di thread un elemento del vettore delle frequenze da espandere. 

L'algoritmo è illustrato in Figura~\ref{pointers_to_index}. 

\begin{figure}[t]
    \centering
	\includegraphics[scale=0.22]{pointers_to_index.png}
	\caption{Pointers to index}
	\label{pointers_to_index}
\end{figure}

Sperimentalmente si nota che il numero ottimale di thread per blocco è $1$.



\section{Struttura dell'implementazione} 
\label{struttura}
	L'implementazione è stata sviluppata utilizzando come supporto il tool \textit{Git}. È stata creata una repository, descritta nella sezione successiva, per poter controllare in modo efficiente lo svilupparsi del progetto. \newline
	
	\noindent Link repo: 	\href{https://github.com/michelepenzo/architetture-avanzate}{github.com/michelepenzo/architetture-avanzate}\newline
	
	\subsection{Struttura delle directory}
	La struttura delle directory è rispecchiata nel seguente schema:
	\dirtree{%
		.1 root.
			.2 doc.
				.3 {report\_aa.$ * $}.				
			.2 code.
				.3 {matrices}.
					.4 {$ *$.mtx }.
				.3 {include}.
					.4 {matrix.hh}.
					.4 {merge\_step.hh}.
					.4 {procedures.hh}.
					.4 {transposers.hh}.
					.4 {Timer.$ * $}.
					.4 {utilities.hh}.
				.3 {src}.
					.4 {all cuda procedures..}.
					.4 {transposer.cu}.
					.4 {main.cu}.
				.3 {test}.	
					.4 {test\_main.cu}.
				.3 Makefile.			
				.3 {timing\_analysis.csv}.			
			.2 README.md.
	}
	\mbox{}

	Il file \texttt{matrix.hh} è composto da due classi con i relativi campi e metodi che si occupa della:
	\begin{enumerate}
		\item costruzione della matrice sparsa in formato Csr: \textit{class~SparseMatrix},
		\item costruzione della matrice densa: \textit{class FullMatrix}.
	\end{enumerate}
	Il file \texttt{procedures.hh} contiene tutte le procedure descritte in sezione~\ref{procedure} ed è diviso in due \texttt{namespace}. Essi riferiscono all'implementazione effettuta, ovvero \textit{cuda} o \textit{reference}(seriale).\newline
	È presente inoltre il file \texttt{transposers.hh} che funge da ``wrapper'' per le quattro implementazioni descritte nella prossima sezione.\newline
	Un'altro file è il \texttt{merge\_step.hh} che contiene tutte le implementazioni del \textit{segmented merge descritto} in \ref{seg-merge}.\newline
	Infine abbiamo il timer e un file che contiene tutte le funzioni utilizzate utili per debug, stampa, allocazione e deallocazione, generazione dei valori random, controllo errori e altro.\newline
	
	All'interno della direcotry \textit{src} abbiamo tutte le implementazioni delle procedure cuda utilizzate nel progetto, oltre che al file \texttt{transposers.cu} che contiene le implementazioni descritte in~\ref{metodologie}.\newline
	Infine è presente il \texttt{main.cu}, che si occupa di eseguire tutte le metodologie implementate.\newline

	È presente inoltre un'ultima directory con relativo file che si occupa della fase di test delle singole componenti.
	
	\subsection{Test delle componenti}	
	Per le singole componenti(scan, sort, index to pointers ...) è stato implementato un'altro eseguibile. Lanciando il comando \textit{make test} viene eseguito l'applicativo che effettuta il test prima per piccole istanze e poi per grandi istanze. Cosi facendo tutte le componenti vengono testate con diversi valori.\newline
	La singola componente del programma viene quindi eseguita sia con la sua implementazione seriale, sia con quella in parallelo. Tramite questa modalità stato più semplice testare le singole componenti e successivamente, dopo aver effettuato dei test complessi ed averli superati a parte, è risultato più semplice unire il tutto per ottenere l'implementazione finale.
	
	\subsection{Applicativo finale}
	\label{applicativo-finale}
	% più matrici testate assieme
	Tramite il comando \textit{make run} viene eseguito l'applicativo finale. Questo esegue per un numero di iterazioni le metodologie implementate. All'interno del tag \texttt{run} nel \textit{Makefile} viene eseguito varie volte l'eseguibile, ogni volta con valori diversi.\newline
	Il file eseguibile può essere eseguito passando:
	\begin{itemize}
		\item un valore (\textit{file.mtx}): esegue le metodolgie su una matrice sparsa caricata da file,
		\item tre valori (\textit{m n nnz}): prende i tre valori classici per la creazione della matrice e genera casualmente la matrice sparsa delle dimensioni richieste,
		\item nessun valore: matrice generate casualmente con valori fissi.
	\end{itemize}
	Alla fine le tempistiche vengono concatenate all'interno del file \textit{timing\_analysis.csv} per effettuarne una migliore lettura.


\begin{figure*}[t]
    \centering
    \scalebox{0.9}{
    \begin{tabular}{lS[table-format=7.0] S[table-format=7.0] S[table-format=8.0] S[table-format=4.2] S[table-format=4.2] S[table-format=4.2] S[table-format=4.2] S[table-format=4.2]}
    \toprule
    \textbf{Nome} 
    & \textbf{M}
    & \textbf{N}
    & \textbf{NNZ}
    & \textbf{Serial}
    & \textbf{ScanTrans}
    & \textbf{MergeTrans}
    & \textbf{\cuSPARSE 1}
    & \textbf{\cuSPARSE 2}
    \\ \midrule
    language.mtx	
        & 399130
        & 399130
        & 1216334
        &  55.75
        & 114.71
        & 197.05
        & 122.02
        &  18.36 \\
    webbase-1M.mtx	
        & 1000005	
        & 1000005
        & 3105536	
        & 138,84	
        & 278,55	
        & 520,90
        & 149,81	
        &  51,00 \\
    rajat21.mtx	
        & 411676	
        & 411676	
        & 1893370	
        &  77,97	
        & 147,64	
        & 306,56	
        & 127,31	
        &  26,77 \\
    ASIC\_680k.mtx	
        & 682862	
        & 682862	
        & 3871773	
        & 154,37	
        & 265,47	
        & 844,76	
        & 155,31	
        &  56,74 \\
    memchip.mtx	
        & 2707524	
        & 2707524	
        & 14810202	
        &  594,18	
        &  994,62	
        & 2328,67	
        &  298,93	
        &  188,07 \\
    cant.mtx	
        & 62451	
        & 62451	
        & 2034917	
        &  73,83	
        & 114,40	
        & 248,49	
        & 127,26	
        &  31,11 \\
    FullChip.mtx	
        & 2987012	
        & 2987012	
        & 26621990	
        &  997,37	
        & 1543,02	
        & 9481,42	
        &  454,53	
        &  328,32 \\
    stomach.mtx	
        & 213360	
        & 213360	
        & 3021648	
        & 110,84	
        & 164,30
        & 387,78	
        & 139,57	
        &  40,26 \\
    web-Google.mtx	
        & 916428	
        & 916428	
        & 5105039	
        &  399,63	
        &  382,37	
        & 3327,56	
        &  170,59	
        &   73,25 \\
    random	
        & 100000	
        & 100000	
        & 10000000	
        &  898,99	
        &  475,17	
        & 3056,12	
        &  210,57	
        &  130,25 \\
    random	
        & 100000	
        & 100000	
        & 10000000	
        &  902,23	
        &  475,72	
        & 3060,25	
        &  208,22	
        &  133,22 \\
    random	
        & 150000	
        & 200000	
        & 5000000	
        &  523,66	
        &  263,90	
        & 1351,83	
        &  161,90	
        &   72,70 \\
    random	
        & 150000	
        & 200000	
        & 5000000	
        &  527,15	
        &  262,88	
        & 1353,93	
        &  165,19	
        &   74,98 \\
    random	
        & 500000	
        & 500000	
        & 10000000	
        & 1380,96	
        &  532,76	
        & 2853,55	
        &  227,33	
        &  141,18 \\ \bottomrule
    \end{tabular}}
    \caption{Risultati sperimentali -  $\textrm{M}, \textrm{N}, \textrm{NNZ}$ rispettivamente numero di righe, di colonne, di elementi non nulli della matrice. I tempi sono in \si{\milli\second}.}
    \label{results}
\end{figure*}

\begin{figure*}[t]
    \centering
    \scalebox{0.9}{
    \begin{tabular}{lS[table-format=7.0] S[table-format=7.0] S[table-format=8.0] S[table-format=4.2] S[table-format=4.2] S[table-format=4.2] S[table-format=4.2] S[table-format=4.2]}
    \toprule
    \textbf{Nome} 
    & \textbf{M}
    & \textbf{N}
    & \textbf{NNZ}
    & \textbf{Serial}
    & \textbf{ScanTrans}
    & \textbf{MergeTrans}
    & \textbf{\cuSPARSE 1}
    & \textbf{\cuSPARSE 2}
    \\ \midrule
    language.mtx	
        & 399130	
        & 399130	
        & 1216334	
        & 1.00
        & 0.49	
        & 0.28	
        & 0.46	
        & 3.04 \\
    webbase-1M.mtx	
        & 1000005	
        & 1000005	
        & 3105536	
        & 1.00
        & 0.50	
        & 0.27	
        & 0.93	
        & 2.72 \\
    rajat21.mtx	
        & 411676	
        & 411676	
        & 1893370	
        & 1.00
        & 0.53	
        & 0.25	
        & 0.61	
        & 2.91 \\
    ASIC\_680k.mtx	
        & 682862	
        & 682862	
        & 3871773	
        & 1.00
        & 0.58	
        & 0.18	
        & 0.99	
        & 2.72 \\
    memchip.mtx	
        & 2707524	
        & 2707524	
        & 14810202	
        & 1.00
        & 0.60	
        & 0.26	
        & 1.99	
        & 3.16 \\
    cant.mtx	
        & 62451	
        & 62451	
        & 2034917	
        & 1.00
        & 0.65	
        & 0.30	
        & 0.58	
        & 2.37 \\
    FullChip.mtx	
        & 2987012	
        & 2987012	
        & 26621990	
        & 1.00
        & 0.65	
        & 0.11	
        & 2.19	
        & 3.04 \\
    stomach.mtx	
        & 213360	
        & 213360	
        & 3021648	
        & 1.00	
        & 0.67	
        & 0.29	
        & 0.79	
        & 2.75 \\
    web-Google.mtx	
        & 916428	
        & 916428	
        & 5105039	
        & 1.00
        & 1.05	
        & 0.12	
        & 2.34	
        & 5.46 \\
    random	
        & 100000	
        & 100000	
        & 10000000	
        & 1.00	
        & 1.89	
        & 0.29	
        & 4.27	
        & 6.90 \\
    random	
        & 100000	
        & 100000	
        & 10000000	
        & 1.00	
        & 1.90	
        & 0.29	
        & 4.33	
        & 6.77 \\
    random
        & 150000	
        & 200000	
        & 5000000	
        & 1.00	
        & 1.98	
        & 0.39	
        & 3.23	
        & 7.20 \\
    random
        & 150000	
        & 200000	
        & 5000000	
        & 1.00	
        & 2.01	
        & 0.39	
        & 3.19	
        & 7.03 \\
    random
        & 500000	
        & 500000	
        & 10000000	
        & 1.00	
        & 2.59	
        & 0.48	
        & 6.07	
        & 9.78 \\ \bottomrule
    \end{tabular}}
    \caption{Risultati sperimentali -  Speedup}
    \label{results_speedup}
\end{figure*}




\begin{figure*}[t]
    \centering
    \scalebox{0.9}{
    \begin{tabular}{lS[table-format=7.0] S[table-format=7.0] S[table-format=8.0] S[table-format=4.2] S[table-format=4.2] S[table-format=4.2] S[table-format=4.2] S[table-format=4.2]}
    \toprule
    \textbf{Nome} 
    & \textbf{M}
    & \textbf{N}
    & \textbf{NNZ}
    & \textbf{Serial}
    & \textbf{ScanTrans}
    & \textbf{MergeTrans}
    & \textbf{\cuSPARSE 1}
    & \textbf{\cuSPARSE 2}
    \\ \midrule
    language.mtx	
        & 399130
        & 399130
        & 1216334
        &  55.75
        & 114.71
        & 197.05
        & 122.02
        &  18.36 \\
    webbase-1M.mtx	
        & 1000005	
        & 1000005
        & 3105536	
        & 138,84	
        & 278,55	
        & 520,90
        & 149,81	
        &  51,00 \\
    rajat21.mtx	
        & 411676	
        & 411676	
        & 1893370	
        &  77,97	
        & 147,64	
        & 306,56	
        & 127,31	
        &  26,77 \\
    ASIC\_680k.mtx	
        & 682862	
        & 682862	
        & 3871773	
        & 154,37	
        & 265,47	
        & 844,76	
        & 155,31	
        &  56,74 \\
    memchip.mtx	
        & 2707524	
        & 2707524	
        & 14810202	
        &  594,18	
        &  994,62	
        & 2328,67	
        &  298,93	
        &  188,07 \\
    cant.mtx	
        & 62451	
        & 62451	
        & 2034917	
        &  73,83	
        & 114,40	
        & 248,49	
        & 127,26	
        &  31,11 \\
    FullChip.mtx	
        & 2987012	
        & 2987012	
        & 26621990	
        &  997,37	
        & 1543,02	
        & 9481,42	
        &  454,53	
        &  328,32 \\
    stomach.mtx	
        & 213360	
        & 213360	
        & 3021648	
        & 110,84	
        & 164,30
        & 387,78	
        & 139,57	
        &  40,26 \\
    web-Google.mtx	
        & 916428	
        & 916428	
        & 5105039	
        &  399,63	
        &  382,37	
        & 3327,56	
        &  170,59	
        &   73,25 \\
    random	
        & 100000	
        & 100000	
        & 10000000	
        &  898,99	
        &  475,17	
        & 3056,12	
        &  210,57	
        &  130,25 \\
    random	
        & 100000	
        & 100000	
        & 10000000	
        &  902,23	
        &  475,72	
        & 3060,25	
        &  208,22	
        &  133,22 \\
    random	
        & 150000	
        & 200000	
        & 5000000	
        &  523,66	
        &  263,90	
        & 1351,83	
        &  161,90	
        &   72,70 \\
    random	
        & 150000	
        & 200000	
        & 5000000	
        &  527,15	
        &  262,88	
        & 1353,93	
        &  165,19	
        &   74,98 \\
    random	
        & 500000	
        & 500000	
        & 10000000	
        & 1380,96	
        &  532,76	
        & 2853,55	
        &  227,33	
        &  141,18 \\ \bottomrule
    \end{tabular}}
    \caption{Risultati sperimentali -  $\textrm{M}, \textrm{N}, \textrm{NNZ}$ rispettivamente numero di righe, di colonne, di elementi non nulli della matrice. I tempi sono in \si{\milli\second}.}
    \label{results}
\end{figure*}

\begin{figure*}[t]
    \centering
    \scalebox{0.9}{
    \begin{tabular}{lS[table-format=7.0] S[table-format=7.0] S[table-format=8.0] S[table-format=4.2] S[table-format=4.2] S[table-format=4.2] S[table-format=4.2] S[table-format=4.2]}
    \toprule
    \textbf{Nome} 
    & \textbf{M}
    & \textbf{N}
    & \textbf{NNZ}
    & \textbf{Serial}
    & \textbf{ScanTrans}
    & \textbf{MergeTrans}
    & \textbf{\cuSPARSE 1}
    & \textbf{\cuSPARSE 2}
    \\ \midrule
    language.mtx	
        & 399130	
        & 399130	
        & 1216334	
        & 1.00
        & 0.49	
        & 0.28	
        & 0.46	
        & 3.04 \\
    webbase-1M.mtx	
        & 1000005	
        & 1000005	
        & 3105536	
        & 1.00
        & 0.50	
        & 0.27	
        & 0.93	
        & 2.72 \\
    rajat21.mtx	
        & 411676	
        & 411676	
        & 1893370	
        & 1.00
        & 0.53	
        & 0.25	
        & 0.61	
        & 2.91 \\
    ASIC\_680k.mtx	
        & 682862	
        & 682862	
        & 3871773	
        & 1.00
        & 0.58	
        & 0.18	
        & 0.99	
        & 2.72 \\
    memchip.mtx	
        & 2707524	
        & 2707524	
        & 14810202	
        & 1.00
        & 0.60	
        & 0.26	
        & 1.99	
        & 3.16 \\
    cant.mtx	
        & 62451	
        & 62451	
        & 2034917	
        & 1.00
        & 0.65	
        & 0.30	
        & 0.58	
        & 2.37 \\
    FullChip.mtx	
        & 2987012	
        & 2987012	
        & 26621990	
        & 1.00
        & 0.65	
        & 0.11	
        & 2.19	
        & 3.04 \\
    stomach.mtx	
        & 213360	
        & 213360	
        & 3021648	
        & 1.00	
        & 0.67	
        & 0.29	
        & 0.79	
        & 2.75 \\
    web-Google.mtx	
        & 916428	
        & 916428	
        & 5105039	
        & 1.00
        & 1.05	
        & 0.12	
        & 2.34	
        & 5.46 \\
    random	
        & 100000	
        & 100000	
        & 10000000	
        & 1.00	
        & 1.89	
        & 0.29	
        & 4.27	
        & 6.90 \\
    random	
        & 100000	
        & 100000	
        & 10000000	
        & 1.00	
        & 1.90	
        & 0.29	
        & 4.33	
        & 6.77 \\
    random
        & 150000	
        & 200000	
        & 5000000	
        & 1.00	
        & 1.98	
        & 0.39	
        & 3.23	
        & 7.20 \\
    random
        & 150000	
        & 200000	
        & 5000000	
        & 1.00	
        & 2.01	
        & 0.39	
        & 3.19	
        & 7.03 \\
    random
        & 500000	
        & 500000	
        & 10000000	
        & 1.00	
        & 2.59	
        & 0.48	
        & 6.07	
        & 9.78 \\ \bottomrule
    \end{tabular}}
    \caption{Risultati sperimentali -  Speedup}
    \label{results_speedup}
\end{figure*}





\section{Risultati sperimentali}

Confrontiamo ora le performance delle varie implementazioni che seguono:
\begin{itemize}
    \item seriale;
    \item parallela \emph{scan trans};
    \item parallela \emph{merge trans};
    \item \cuSPARSE (2).
\end{itemize}

Le istanze su cui vengono eseguiti i vari algoritmi sono in parte generate in modo casuale (a partire dalle specifiche della matrice sparsa), in parte recuperate dal dataset "University  of Florida sparse  matrix collection`` \cite{dataset}. Tale dataset è stato usato per valutare le performance degli algoritmi in \cite{parallelTrans}.

La macchina sul quale vengono eseguiti i vari algoritmi è equipaggiata con una scheda NVidia GeForce GTX 780 con Cuda Runtime 10.2.

I risultati sono visibili in Tabella~\ref{results}. 

\section{Considerazioni finali}\label{conclusioni}
	% tirare le somme di cosa abbiamo ottenuto
	% cosa migliorare?
	% come possiamo continuare il progetto se avessimo avuto più tempo?
		% algo merge con merge buffer che è gia implementato
		% testare l'efficienza delle componenti rispetto alla implementazione nvidia
		% idx to pntrs (soffre race condition) --> un thread ha un blocco
		

\bibliographystyle{IEEEtran}
\bibliography{biblio}


%\appendix
%Se non avete abbastanza spazio, potete inserire le figure delle EFSM in una  pagina extra, appendice. Un esempio di come potete fare solo le Figure~\ref{fig:grande}, \ref{fig:piccola1}, \ref{fig:piccola2}.

\end{document}