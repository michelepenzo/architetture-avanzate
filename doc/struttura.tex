
\section{Struttura dell'implementazione} 
\label{struttura}
	L'implementazione è stata sviluppata utilizzando come supporto il tool \textit{Git}. È stata creata una repository, descritta nella sezione successiva, per poter controllare in modo efficiente lo svilupparsi del progetto. \newline
	
	\noindent Link repo: 	\href{https://github.com/michelepenzo/architetture-avanzate}{github.com/michelepenzo/architetture-avanzate}\newline
	
	\subsection{Struttura delle directory}
	La struttura delle directory è rispecchiata nel seguente schema:
	\dirtree{%
		.1 root.
			.2 doc.
				.3 {report\_aa.$ * $}.				
			.2 code.
				.3 {matrices}.
					.4 {$ *$.mtx }.
				.3 {include}.
					.4 {matrix.hh}.
					.4 {merge\_step.hh}.
					.4 {procedures.hh}.
					.4 {transposers.hh}.
					.4 {Timer.$ * $}.
					.4 {utilities.hh}.
				.3 {src}.
					.4 {all cuda procedures..}.
					.4 {transposer.cu}.
					.4 {main.cu}.
				.3 {test}.	
					.4 {test\_main.cu}.
				.3 Makefile.			
				.3 {timing\_analysis.csv}.			
			.2 README.md.
	}
	\mbox{}

	Il file \texttt{matrix.hh} è composto da due classi con i relativi campi e metodi che si occupa della:
	\begin{enumerate}
		\item costruzione della matrice sparsa in formato Csr: \textit{class~SparseMatrix},
		\item costruzione della matrice densa: \textit{class FullMatrix}.
	\end{enumerate}
	Il file \texttt{procedures.hh} contiene tutte le procedure descritte in sezione~\ref{procedure} ed è diviso in due \texttt{namespace}. Essi riferiscono all'implementazione effettuta, ovvero \textit{cuda} o \textit{reference}(seriale).\newline
	È presente inoltre il file \texttt{transposers.hh} che funge da ``wrapper'' per le quattro implementazioni descritte nella prossima sezione.\newline
	Un'altro file è il \texttt{merge\_step.hh} che contiene tutte le implementazioni del \textit{segmented merge descritto} in \ref{seg-merge}.\newline
	Infine abbiamo il timer e un file che contiene tutte le funzioni utilizzate utili per debug, stampa, allocazione e deallocazione, generazione dei valori random, controllo errori e altro.\newline
	
	All'interno della direcotry \textit{src} abbiamo tutte le implementazioni delle procedure cuda utilizzate nel progetto, oltre che al file \texttt{transposers.cu} che contiene le implementazioni descritte in~\ref{metodologie}.\newline
	Infine è presente il \texttt{main.cu}, che si occupa di eseguire tutte le metodologie implementate.\newline

	È presente inoltre un'ultima directory con relativo file che si occupa della fase di test delle singole componenti.
	
	\subsection{Test delle componenti}	
	Per le singole componenti(scan, sort, index to pointers ...) è stato implementato un'altro eseguibile. Lanciando il comando \textit{make test} viene eseguito l'applicativo che effettuta il test prima per piccole istanze e poi per grandi istanze. Cosi facendo tutte le componenti vengono testate con diversi valori.\newline
	La singola componente del programma viene quindi eseguita sia con la sua implementazione seriale, sia con quella in parallelo. Tramite questa modalità stato più semplice testare le singole componenti e successivamente, dopo aver effettuato dei test complessi ed averli superati a parte, è risultato più semplice unire il tutto per ottenere l'implementazione finale.
	
	\subsection{Applicativo finale}
	\label{applicativo-finale}
	% più matrici testate assieme
	Tramite il comando \textit{make run} viene eseguito l'applicativo finale. Questo esegue per un numero di iterazioni le metodologie implementate. All'interno del tag \texttt{run} nel \textit{Makefile} viene eseguito varie volte l'eseguibile, ogni volta con valori diversi.\newline
	Il file eseguibile può essere eseguito passando:
	\begin{itemize}
		\item un valore (\textit{file.mtx}): esegue le metodolgie su una matrice sparsa caricata da file,
		\item tre valori (\textit{m n nnz}): prende i tre valori classici per la creazione della matrice e genera casualmente la matrice sparsa delle dimensioni richieste,
		\item nessun valore: matrice generate casualmente con valori fissi.
	\end{itemize}
	Alla fine le tempistiche vengono concatenate all'interno del file \textit{timing\_analysis.csv} per effettuarne una migliore lettura.