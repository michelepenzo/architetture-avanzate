
\section{Struttura dell'implementazione}\label{struttura}

L'intera implementazione è scaricabile attraverso \emph{git} dalla repository \url{https://github.com/michelepenzo/architetture-avanzate}.

La struttura delle directory del progetto è presente in Figura~\ref{fig:struct}. 

La sottodirectory \emph{doc} contiene questo stesso documento in formato \emph{pdf} ed i rispettivi sorgenti \emph{tex}. 

La sottodirectory \emph{code} contiene i sorgenti dell'applicativo principale e di quello secondario di test. 

Lo scopo del primo è generare un file \emph{csv} contente le tempistiche e gli speedup di ogni algoritmo applicato sulle istanze di matrici in input generate casualmente oppure lette da file \emph{mtx} (\emph{market matrix}, una rappresentazione della matrice sparsa in formato COO attraverso file di testo). 

Lo scopo del secondo applicativo è testare il corretto funzionamento di ogni componente del progetto. In particolare, sottopongo le stesse istanze di array o matrici (a seconda del componente che sto testando) sia alla funzione che ne implementa l'algoritmo parallelo, sia alla funzione che ne implementa l'algoritmo seriale. Mi aspetto che i risultati siano uguali per tutte le istanze.

\begin{figure}
    \dirtree{%
	.1 root.		
	    .2 README.md.
		.2 doc.			
		.2 code.
			.3 {matrices}.
			.3 {include}.
				.4 {matrix.hh}.
				.4 {merge\_step.hh}.
				.4 {procedures.hh}.
				.4 {transposers.hh}.
				.4 {Timer.$ * $}.
				.4 {utilities.hh}.
			.3 {src}.
				.4 {...}.
				.4 {transposer.cu}.
				.4 {main.cu}.
			.3 {test}.	
				.4 {...}.
				.4 {tester.hh}.
				.4 {test\_main.cu}.
			.3 Makefile.			
			.3 {timing\_analysis.csv}.	
}
    \caption{Struttura delle directory del progetto}
    \label{fig:struct}
\end{figure}

In particolare:
\begin{itemize}
    \item il file \texttt{include/matrix.hh} contiene le classi \emph{FullMatrix} e \emph{SparseMatrix} che si occupano di allocare nella memoria host lo spazio necessario a contenere la matrice date le sue specifiche ($m$, $n$, $nnz$), sia come matrice estesa sia in formato \emph{csr}. Inoltre contiene i metodi per inizializzare la matrice in modo casuale e per passare da un formato all'altro;
    \item il file \texttt{include/utils.hh} contiene i metodi di utilità quali le funzioni di stampa e di allocazione e deallocazione della memoria device;
    \item i file \texttt{include/procedures.hh} e \texttt{merge\_step.hh} contengono le dichiarazioni delle procedure descritte nella Sezione~\ref{procedure}. La maggior parte delle definizioni sono presenti nella sottodirectory \texttt{src}, nel caso del metodo \emph{merge\_step} la definizione è scritta direttamente nell'header. Questa scelta è conveniente in quanto tale funzione è definita rispetto ad un tipo generico, se la definizione fosse stata riportata nei file cpp avremmo dovuto indicare esplicitamente i tipi concreti per il quale vogliamo rendere disponibile il nostro metodo (\cite{template});
    \item il file \texttt{include/transposers.hh} e rispettivo sorgente \texttt{src/transposers.cu} contengono le dichiarazioni e definizioni dei metodi che effettuano la trasposta: seriale, parallela con \ScanTrans{} e \MergeTrans{} ed infine da libreria \cuSPARSE{} con entrambi i possibili algoritmi;
    \item i file \texttt{Timer.*} contengono una classe di utilità \emph{timer} che permette di cronometrare il tempo che occorre per eseguire un dato pezzo di codice;
    \item il file \texttt{src/main.cu} contiene l'applicativo principale che chiama i diversi metodi di trasposta sulla matrice fornita in input, ne cronometra le tempistiche e le stampa in output;
    \item il file \texttt{test/tester.hh} contiene la classe astratta \emph{tester} che espone un metodo \emph{test\_many\_instances} che chiama un metodo astratto \emph{test} un numero arbitrario di volte, passandogli in input un intero che rappresenta il numero dell'istanza attuale che può essere usato per decidere la dimensione dell'istanza di test. Attualmente le istanze testate vanno da 1 a 20'000, poi da 20'000 a 20'000'000 con step $\times 1.5$; 
    \item i file \texttt{test/tester\_*.hh} si occupano di testare un singolo componente, contengono ciascuna uno o più classi concrete che estendono la classe astratta \emph{tester};
    \item il file \texttt{test/test\_main.cu} contiene l'applicativo di test che alloca oggetti delle varie classi tester, li avvia e ne stampa gli eventuali errori a video.
\end{itemize}
	
\section{Avvio degli applicativi}

L'applicativo principale può essere avviato con tre modalità diverse:
\begin{itemize}
    \item senza parametri, genera una matrice $500'000\times 500'000$ con $10'000'000$ elementi non nulli, ne valuta le tempistiche con i diversi algoritmi ritornando la media su un numero arbitrario di esecuzioni;
    \item con tre parametri interi $\mathrm{m}$, $\mathrm{n}$, $\mathrm{nnz}$, genera una matrice avente le dimensioni ricevute in input e procede alla valutazione delle tempistiche come sopra;
    \item con un parametro stringa \texttt{filename}, legge da file una matrice che deve essere nel formato \emph{mtx market matrix}.
\end{itemize}

Avviando l'applicativo principale attraverso il Makefile con \emph{make run} viene avviato molteplici volte l'applicativo principale, ogni volta con un'istanza di matrice diversa in input, con lo scopo di popolare un documento \emph{timing\_analysis.csv} contenente le tempistiche medie su diversi input. 

L'applicativo di test di avvia con una sola modalità equivalentemente avviando il nome dell'applicativo senza parametro oppure attraverso il Makefile con \emph{make test}. Su \emph{stdout} viene stampato ``no" se almeno un test ha mostrato anomalie, ``ok" altrimenti. 