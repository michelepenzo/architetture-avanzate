\begin{figure*}[t]
    \centering
    \scalebox{0.9}{
    \begin{tabular}{lS[table-format=7.0] S[table-format=7.0] S[table-format=8.0] S[table-format=4.2] S[table-format=4.2] S[table-format=4.2] S[table-format=4.2] S[table-format=4.2]}
    \toprule
    \textbf{Nome} 
    & \textbf{M}
    & \textbf{N}
    & \textbf{NNZ}
    & \textbf{Serial}
    & \textbf{\ScanTrans{}}
    & \textbf{\MergeTrans{}}
    & \textbf{\cuSPARSE{} 1}
    & \textbf{\cuSPARSE{} 2}
    \\ \midrule
    language.mtx	
        & 399130
        & 399130
        & 1216334
        &  55.75
        & 114.71
        & 197.05
        & 122.02
        &  18.36 \\
    webbase-1M.mtx	
        & 1000005	
        & 1000005
        & 3105536	
        & 138,84	
        & 278,55	
        & 520,90
        & 149,81	
        &  51,00 \\
    rajat21.mtx	
        & 411676	
        & 411676	
        & 1893370	
        &  77,97	
        & 147,64	
        & 306,56	
        & 127,31	
        &  26,77 \\
    ASIC\_680k.mtx	
        & 682862	
        & 682862	
        & 3871773	
        & 154,37	
        & 265,47	
        & 844,76	
        & 155,31	
        &  56,74 \\
    memchip.mtx	
        & 2707524	
        & 2707524	
        & 14810202	
        &  594,18	
        &  994,62	
        & 2328,67	
        &  298,93	
        &  188,07 \\
    cant.mtx	
        & 62451	
        & 62451	
        & 2034917	
        &  73,83	
        & 114,40	
        & 248,49	
        & 127,26	
        &  31,11 \\
    FullChip.mtx	
        & 2987012	
        & 2987012	
        & 26621990	
        &  997,37	
        & 1543,02	
        & 9481,42	
        &  454,53	
        &  328,32 \\
    stomach.mtx	
        & 213360	
        & 213360	
        & 3021648	
        & 110,84	
        & 164,30
        & 387,78	
        & 139,57	
        &  40,26 \\
    web-Google.mtx	
        & 916428	
        & 916428	
        & 5105039	
        &  399,63	
        &  382,37	
        & 3327,56	
        &  170,59	
        &   73,25 \\
    random	
        & 100000	
        & 100000	
        & 10000000	
        &  898,99	
        &  475,17	
        & 3056,12	
        &  210,57	
        &  130,25 \\
    random	
        & 100000	
        & 100000	
        & 10000000	
        &  902,23	
        &  475,72	
        & 3060,25	
        &  208,22	
        &  133,22 \\
    random	
        & 150000	
        & 200000	
        & 5000000	
        &  523,66	
        &  263,90	
        & 1351,83	
        &  161,90	
        &   72,70 \\
    random	
        & 150000	
        & 200000	
        & 5000000	
        &  527,15	
        &  262,88	
        & 1353,93	
        &  165,19	
        &   74,98 \\
    random	
        & 500000	
        & 500000	
        & 10000000	
        & 1380,96	
        &  532,76	
        & 2853,55	
        &  227,33	
        &  141,18 \\ \bottomrule
    \end{tabular}}
    \caption{Risultati sperimentali -  $\textrm{M}, \textrm{N}, \textrm{NNZ}$ rispettivamente numero di righe, di colonne, di elementi non nulli della matrice. I tempi sono in \si{\milli\second}.}
    \label{results}
\end{figure*}

\begin{figure*}[t]
    \centering
    \scalebox{0.9}{
    \begin{tabular}{lS[table-format=7.0] S[table-format=7.0] S[table-format=8.0] S[table-format=4.2] S[table-format=4.2] S[table-format=4.2] S[table-format=4.2] S[table-format=4.2]}
    \toprule
    \textbf{Nome} 
    & \textbf{M}
    & \textbf{N}
    & \textbf{NNZ}
    & \textbf{Serial}
    & \textbf{\ScanTrans{}}
    & \textbf{\MergeTrans{}}
    & \textbf{\cuSPARSE{} 1}
    & \textbf{\cuSPARSE{} 2}
    \\ \midrule
    language.mtx	
        & 399130	
        & 399130	
        & 1216334	
        & 1.00
        & 0.49	
        & 0.28	
        & 0.46	
        & 3.04 \\
    webbase-1M.mtx	
        & 1000005	
        & 1000005	
        & 3105536	
        & 1.00
        & 0.50	
        & 0.27	
        & 0.93	
        & 2.72 \\
    rajat21.mtx	
        & 411676	
        & 411676	
        & 1893370	
        & 1.00
        & 0.53	
        & 0.25	
        & 0.61	
        & 2.91 \\
    ASIC\_680k.mtx	
        & 682862	
        & 682862	
        & 3871773	
        & 1.00
        & 0.58	
        & 0.18	
        & 0.99	
        & 2.72 \\
    memchip.mtx	
        & 2707524	
        & 2707524	
        & 14810202	
        & 1.00
        & 0.60	
        & 0.26	
        & 1.99	
        & 3.16 \\
    cant.mtx	
        & 62451	
        & 62451	
        & 2034917	
        & 1.00
        & 0.65	
        & 0.30	
        & 0.58	
        & 2.37 \\
    FullChip.mtx	
        & 2987012	
        & 2987012	
        & 26621990	
        & 1.00
        & 0.65	
        & 0.11	
        & 2.19	
        & 3.04 \\
    stomach.mtx	
        & 213360	
        & 213360	
        & 3021648	
        & 1.00	
        & 0.67	
        & 0.29	
        & 0.79	
        & 2.75 \\
    web-Google.mtx	
        & 916428	
        & 916428	
        & 5105039	
        & 1.00
        & 1.05	
        & 0.12	
        & 2.34	
        & 5.46 \\
    random	
        & 100000	
        & 100000	
        & 10000000	
        & 1.00	
        & 1.89	
        & 0.29	
        & 4.27	
        & 6.90 \\
    random	
        & 100000	
        & 100000	
        & 10000000	
        & 1.00	
        & 1.90	
        & 0.29	
        & 4.33	
        & 6.77 \\
    random
        & 150000	
        & 200000	
        & 5000000	
        & 1.00	
        & 1.98	
        & 0.39	
        & 3.23	
        & 7.20 \\
    random
        & 150000	
        & 200000	
        & 5000000	
        & 1.00	
        & 2.01	
        & 0.39	
        & 3.19	
        & 7.03 \\
    random
        & 500000	
        & 500000	
        & 10000000	
        & 1.00	
        & 2.59	
        & 0.48	
        & 6.07	
        & 9.78 \\ \bottomrule
    \end{tabular}}
    \caption{Risultati sperimentali -  Speedup}
    \label{results_speedup}
\end{figure*}





\section{Risultati sperimentali}

Confrontiamo ora le performance delle varie implementazioni che seguono:
\begin{itemize}
    \item seriale;
    \item parallela \emph{scan trans};
    \item parallela \emph{merge trans};
    \item \cuSPARSE (2).
\end{itemize}

Le istanze su cui vengono eseguiti i vari algoritmi sono in parte generate in modo casuale (a partire dalle specifiche della matrice sparsa), in parte recuperate dal dataset "University  of Florida sparse  matrix collection`` \cite{dataset}. Tale dataset è stato usato per valutare le performance degli algoritmi in \cite{parallelTrans}.

La macchina sul quale vengono eseguiti i vari algoritmi è equipaggiata con una scheda NVidia GeForce GTX 780 con Cuda Runtime 10.2.

I risultati sono visibili in Tabella~\ref{results}. 

\section{Considerazioni finali}\label{conclusioni}
Possiamo notare come le implementazioni citate in \cite{parallelTrans} e da noi sviluppate non siano all'altezza delle versioni di \cuSPARSE.

Contrariamente a quanto asserito in \cite{parallelTrans}, nel nostro caso \textit{Scan Trans} si comporta meglio di \textit{Merge Trans}. Questo potrebbe essere dovuto all'implementazione da noi usata nel merge spiegato in sezione \ref{merge} e dalle diverse ottimizzazioni utilizzate nell'implementazione del paper della quale non ne siamo a conoscenza.

Abbiamo inoltre notato come \ScanTrans ottenga risultati migliori se eseguito su matrici ``random'' dove i valori, a differenza delle matrici in formato \texttt{.mtx}, sono interi e non decimali.

Come possibili future implementazioni per migliorare l'efficienza del progetto abbiamo pensato come il package \textit{modern gpu} prenste su Github ci possa tornare utile. Esso mette a disposione implementazioni di alcuni componenti a noi utili per l'obiettivo finale. A partire da queste implementazioni avremmo potuto confrontare le componenti da noi sviluppate con quelle presenti per capire dove migliorare. 



