

\section{Risultati sperimentali}

Confrontiamo ora le performance delle varie implementazioni che seguono:
\begin{itemize}
    \item seriale;
    \item parallela \emph{scan trans};
    \item parallela \emph{merge trans};
    \item \cuSPARSE (entrambi gli algoritmi).
\end{itemize}

Le istanze su cui vengono eseguiti i vari algoritmi sono in parte generate in modo casuale (a partire dalle specifiche della matrice sparsa), in parte recuperate dal dataset "University  of Florida sparse  matrix collection`` \cite{dataset}. Tale dataset è stato usato per valutare le performance degli algoritmi in \cite{parallelTrans}.

La macchina sul quale vengono eseguiti i vari algoritmi è equipaggiata con una scheda NVidia GeForce GTX 780 con Cuda Runtime 10.2.

I risultati sono visibili in Tabella~\ref{results}. 

\section{Considerazioni finali}\label{conclusioni}
	% tirare le somme di cosa abbiamo ottenuto
	% cosa migliorare?
	% come possiamo continuare il progetto se avessimo avuto più tempo?
		% algo merge con merge buffer che è gia implementato
		% testare l'efficienza delle componenti rispetto alla implementazione nvidia
		% idx to pntrs (soffre race condition) --> un thread ha un blocco